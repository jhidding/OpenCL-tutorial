\documentclass{article}

\usepackage{acronym}
\usepackage{minted}
\usepackage{fontspec}
\usepackage{hyperref}
\usepackage{xcolor}

\definecolor{shellbg}{rgb}{0.9,0.9,0.9}

\hypersetup{
  colorlinks,
  linkcolor={red!50!black},
  citecolor={blue!50!black},
  urlcolor={blue!80!black}
}

\begin{document}

  \section{Introduction}
    Since you're reading this, I asume you want to learn about programming \acf{OpenCL}. This tutorial is not going to teach you \emph{why} you'd want to program using \ac{OpenCL}. I can only tell you why \emph{I} want to program \ac{OpenCL}, then you can decide if our goals are in alignment: I want to do generic number crunching, using the speed of \acp{GPU} where possible.

    The second question you could (but shouldn't) ask is: why not \acs{CUDA}? nVidia's \ac{CUDA} is a proprietary system that only runs on nVidia hardware. When I write software I don't want its use being limited to machines with an nVidia \ac{GPU}. If all we have is a multi-core \acs{CPU}, the software should run all the same.

    During this tutorial we'll develop ways of embedding the \ac{OpenCL} enabled code into other code without having to write too much boiler plate at every time we're calling into \ac{OpenCL}. This means we'll be using C++. All the extra utilities that we use on top of the C++ interface to \ac{OpenCL} will be explained in detail.

    \subsection{What you need}
    It is strongly advised to run this tutorial on a distribution of \acs{GNU} running on the Linux kernel. Most of this should work fine on Windows, but this has not been tested.

    \subsubsection{To run \acs{OpenCL}.}
    The \acf{OpenCL} is a specification of an interface for doing computations on heterogenious hardware. To use \ac{OpenCL} you need an actual implementation. The implementation you use will depend on the hardware that you're running on. These implementations are distributed in the form of \acfp{ICD}. Some of these \acp{ICD} are packaged by \acs{GNU}/Linux distributions, but for best performance on larger \acfp{GPU}, it is not unusual that you need to download a non-free \ac{ICD} from the vendor's website.

    \begin{description}
      \item[Intel and \acs{AMD} \acsp{CPU}] Intel ships an \acs{SDK} for which you need to register: \url{https://software.intel.com/en-us/intel-opencl/}. If you don't like this, the alternative is to use the \acs{AMD} driver: \url{http://developer.amd.com/tools-and-sdks/opencl-zone/}, or the open source \acf{PoCL}: \url{http://portablecl.org/}.
      \item[Intel \acs{GPU}] Many laptops come with onboard Intel graphics cards. There is an open source \acs{ICD} for Intel \acp{GPU} called Beignet: \url{https://01.org/beignet}. Beignet currently supports OpenCL 2.0 for Skylake \acp{GPU} and later, and is available all major \acs{GNU}/Linux distributions.
      \item[Nvidia \acsp{GPU}] Nvidia has an \acs{ICD}, but it only supports OpenCL 1.2. Most people will tell you: just use \ac{CUDA}. At some point Nvidia will have to support Vulkan, and Vulkan and \ac{OpenCL} have the same \ac{SPIR} back-end.
      \item[\acs{AMD} \acsp{GPU}] \acs{AMD} has the most complete support for \ac{OpenCL}. The \ac{SDK} can be found at \url{http://developer.amd.com/tools-and-sdks/opencl-zone/}.
    \end{description}

  The nice thing about \acsp{ICD}, is that you can have multiple \ac{OpenCL} implementations installed on the same system. Make sure you install an \ac{ICD} loader and the develoment headers too.

  \subsubsection{To run this tutorial.}
  Since we're going to use the C++ interface to the \ac{OpenCL} host \acs{API}, you'll also need the {\tt cl2.hpp} file. Not all distributions ship this, so a version is included in this repository. We'll be working on \ac{PNG} images using the {\tt png++} library. This is also included in this repository, but it talks to {\tt libpng}, which you need to have installed.

  Lessons in this tutorial are built with the Meson build system, which in turn uses Ninja. Meson is written in Python and can be installed using:

  \begin{minted}[bgcolor=shellbg]{text}
  $ pip install --user meson
  \end{minted}

  If Ninja is not present on your system, install it by cloning it with {\tt git}, and put the binary in a findable place (this assumes {\tt \~\ /.local/bin} is in your {\tt \$PATH}):

  \begin{minted}[bgcolor=shellbg]{text}
  $ git clone git@github.com:ninja-build/ninja.git
  $ cd ninja
  ninja$ ./configure.py --bootstrap
  ninja$ cp ninja ~/.local/bin
  \end{minted}

  Try building the tutorials now:

  \begin{minted}[bgcolor=shellbg]{text}
  OpenCL-tutorial$ meson build
  The Meson build system
  Version: 0.39.1
  Source dir: <...>/OpenCL-tutorial
  Build dir: <...>/OpenCL-tutorial/build
  ...
  Build targets in project: 6

  OpenCL-tutorial$ cd build && ninja
  [16/16] Linking target lesson3
  \end{minted}

  If all went well, you should now be able to run `cl-info`, on my laptop it finds two platforms:

  \begin{minipage}{\linewidth}
    \begin{minted}[bgcolor=shellbg]{text}
  OpenCL-tutorial/build$ ./cl-info
  Number of platforms: 2

  == Nº 1 =======================================
    Name:      Intel Gen OCL Driver
    Vendor:    Intel
    # devices: 1

    -- Nº 1 -------------------------------------
      Name:             Intel(R) HD Graphics ...
      Device version:   OpenCL 1.2 beignet 1.3
      Driver version:   1.3
      OpenCL version:   OpenCL C 1.2 beignet 1.3
      # Compute units:  24

  == Nº 2 =======================================
    Name:      Portable Computing Language
    Vendor:    The pocl project
    # devices: 1

    -- Nº 1 -------------------------------------
      Name:             pthread-Intel(R) ...
      Device version:   OpenCL 2.0 pocl ...
      Driver version:   0.14-pre
      OpenCL version:   OpenCL C 2.0
      # Compute units:  4
  \end{minted}
  \end{minipage}

  Next, we'll see how we actually find these devices.

  \section{Lesson 1: Finding devices, \acs{OpenCL} architecture}

  \section{Lesson 2: Writing kernels, using buffers and queues}

  \section{Acronyms}
  \begin{acronym}
    \acro{AMD}{Advanced Micro Devices, Inc.}
    \acro{API}{Application Programming Interface}
    \acro{CPU}{Central Processing Unit}
    \acro{CUDA}{Compute Unified Device Architecture}
    \acro{ICD}{Installable Client Driver}
    \acro{GNU}{GNU's Not Unix}
    \acro{GPU}{Graphical Processing Unit}
    \acro{OpenCL}{Open Computing Language}
    \acro{PNG}{Portable Network Graphics}
    \acro{PoCL}{Portable Computing Language}
    \acro{SDK}{Software Development Kit}
    \acro{SPIR}{Standard Portable Intermediate Representation}
  \end{acronym}
\end{document}

% vim: ts=2 sw=2
